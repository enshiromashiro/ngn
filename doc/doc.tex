\documentclass[a4j]{jsarticle}

\usepackage{listings, jlisting}
\lstset{%
  language={HTML},
  basicstyle={\small\ttfamily},%
  identifierstyle={\small},%
  commentstyle={\small\itshape},%
  keywordstyle={\small\bfseries},%
  ndkeywordstyle={\small},%
  stringstyle={\small\ttfamily},
  frame={tlrb},
  breaklines=true,
  columns=[l]{fullflexible},%
  numbers=left,%
  xrightmargin=0zw,%
  xleftmargin=3zw,%
  numberstyle={\scriptsize},%
  stepnumber=1,
  numbersep=1zw,%
  lineskip=-0.5ex%
}

\newcommand{\emptyline}{\vspace{1em}}


\begin{document}

\title{ngn -novel page generator-}
\author{@subaru45}
\date{2015-02-22}
\maketitle



\section{はじめに}

このドキュメントは、小説ページ生成ツールngnのマニュアルです。

ngnが何であるか(\ref{sec:abstract}節)や、ngnの導入方法(\ref{sec:installation}節)、使い方(\ref{sec:howtouse}節)、変換形式の追加方法(\ref{sec:renderer}節)が書かれています。
ngnのGUI版であるngnfの使い方や導入方法は書いていません。



\section{ngnの概要}
\label{sec:abstract}

\subsection{ngnってなんぞや}
ngnはテキストファイルとテンプレートから、HTMLファイルやpixiv小説形式のテキストファイルを生成する、簡単なテンプレートエンジンです。
テキストファイル中の部分に名前を付け(タグ付け)、タグ付けした部分をテンプレートファイル中に埋め込むことができます。

またngn独自の記法を用いてルビや太字、見出しや引用を最終的な形式(HTML、pixiv形式、などなど……)に依存しない形で記述しておき、テンプレートへの埋め込み時に最終的な形式での記法に変換することができます。

言葉ではわかりにくいと思うので、\ref{subsec:example}節にて例を出します。


\subsection{だれ向けなのか}

一つのテキストファイルから、複数の形式にテンプレートを使って書き出したい人向けです。

たとえば、書いた小説データからウェブサイト用のHTML、pixiv用のテキストデータなどをコピペを駆使してつくって「めんどくせー!!」と思っている人には向いています。
あるいは、書き終わってから形式が決まる——最初からフォーマットが決まっていないという人にも向いているかもしれません。

逆に、常に一つの形式しか利用しない人、つまり、pixivにしか投稿しないので小説データにpixiv形式をベタ書きしている人などには、メリットはないかもしれません。


\subsection{利用例}
\label{subsec:example}

利用例を出します。
下記に示すように、ngn形式でテキストファイルを書いたとします。

\lstinputlisting[caption=sample.txt]{sample/sample.txt}

また、テンプレートとして次のようなファイルを用意しておきます。

\lstinputlisting[caption=template.html]{sample/template.html}

ここでコマンドラインで(ウィンドウで操作できるようにする予定です)

\verb|  $ ngn template.html sample.txt output.html|

と叩くと、こんなファイルができあがります。

\lstinputlisting[caption=output.html]{sample/output.html}

output.htmlの中身を見ると、sample.txtの内容がtemplate.htmlの中に埋め込まれていますね。
ngnはこんなことをコピペを駆使して手でやらなくてもよくするツールです。


\subsection{その他}

開発目的はウェブサイトの小説・文章ページを自動生成することですが、ほかの用途にも使えなくもないです。

出力形式はテンプレートファイルの拡張子で判別します。
rendererフォルダに`[拡張子]-renderer.lisp`のファイルがある拡張子のファイルだけ、出力形式として利用できます。

このrendererファイルで、各拡張子のファイル変換の内容を決定しています。
これは独自に追加可能ですが、Common Lispの知識が必要です。
変換処理と追加方法については\ref{sec:renderer}節に記載していますが、難易度は高いです。
作者までまで要望を寄せていだだければ、対応するかもしれません。


\subsection{ライセンス}

ngn は NYSL (煮るなり焼くなり好きにしろライセンス)\footnote{http://nysl.com} を採用しています。
NYSL の元で許される限りにおいて、如何様にも改変し配布し破棄することができます。


\section{導入}
\label{sec:installation}

プラットフォームによって導入方法は違います。

\begin{itemize}
\item Windows ユーザへ\\
  bitbucket の下記ダウンロードページ \\
  http://bitbucket.org/subaru45/ngn/downloads \\
  から``win''の文字のついた一番新しいバージョンのファイルをダウンロードして、好きな場所に解凍してください。
  ngnコマンドはコマンドライン上で動作するアプリケーションなので、コマンドライン操作の基本を知っている必要があります。
  が、今後ウィンドウで操作できるようにする予定なので、少々おまちください。\\

\item OSX ユーザへ\\
  作者はMacを持っていないので、バイナリの提供ができません。
  Wineでなんとかならないかなあ。\\

\item *nix ユーザへ\\
  Ubuntu版バイナリなら提供できますが、そもそもいるのだろうか……。
  いちおうWineで動くとは思います。

\end{itemize}



\section{使い方}
\label{sec:howtouse}

ngn の使い方を説明します。

おおまかには以下のような流れで使います。

\begin{enumerate}
  \item 入力ファイルを用意する \label{enum:input}
  \item テンプレートファイルを用意する \label{enum:template}
  \item ngnコマンドを実行して、出力ファイルを生成する \label{enum:command}
\end{enumerate}

この節では、上の各ステップをよりくわしく説明していきます。

\ref{enum:input}.では入力するためのテキストファイルの書き方を説明します。
テキストデータの部分に対する名前の付け方、テキストデータ中で使える特殊な記法についてくわしく書きます。

\ref{enum:template}.では、出力ファイルを生成するのに必要なテンプレートファイルの書き方を説明します。

\ref{enum:command}.では、ngnコマンドを実行し、出力ファイルを生成する方法を説明します。
コマンドラインでの ngn の操作方法について述べています。

ngnf(ウィンドウでngnを操作するツール)を使う人は、\ref{enum:input}.と\ref{enum:template}.だけ読んでください。
入力ファイルの書き方と、テンプレートファイルの書き方が説明されています。
\ref{enum:command}.を読まなくていいのは、nanfには必要のない少々専門的な説明がでてくるからです。


\subsection{入力ファイルを用意する}

まず、入力するテキストファイルをつくります。

入力するテキストファイルには\textbf{タグ}というものを付ける必要があります。
「タグを\textit{付ける}」と言いましたが、これはテキストファイルの部分に名前を付けることを言います。

はじめに例を見てみましょう。

\begin{lstlisting}[caption=ngnタグ]
:title 題名を書きます
:author 書いた人太郎

:body

 本文を書きます。
; この行はコメントです。無視されます。
 本文の二行目です。

:memo:

* 本文を書く
  * コメントを書く

* あとがきを書く
* それぞれふたつずつ書くこと

:postscript

 あとがきを書きます。
 あとがき二行目。
 あとがきおわり。
\end{lstlisting}

コロン(:)から始まっている行がngnのタグです。
タグは行先頭にコロン(\verb|:|)がきて、その後ろに名前がきます。
名前は好きなように決めることができ、あとで名前をテンプレートファイル中で指定することで、テンプレートファイルに埋め込まれます。
名前に使用できる文字は、小文字のアルファベット(a-z)と数字(0-9)とハイフン(\verb|-|)だけです。

後に\ref{subsubsec:ignore}節でまた説明しますが、\underline{タグのない行は無視}されます。

タグには以下の2種類があります。

\begin{enumerate}
  \item 一行分のテキストに名前を付けるタグ \label{enum:onelinetag}
  \item 複数行分のテキストに名前を付けるタグ \label{enum:blocktag}
\end{enumerate}

上の例でいうと「:title」「:author」の行が1.に、「:memo:」「:postscript」の行が2.に当てはまります。
それぞれ説明していきましょう。


\subsubsection{一行タグ}

まず\ref{enum:onelinetag}.の「一行分のテキストに名前を付けるタグ」について説明します。
長いのでこれを\textbf{一行タグ}と呼ぶことにします。

一行タグは、改行を含まないテキストデータに名前を付けるためのタグです。
たとえばテキストデータの題名、作者名、作成日時、などに名前を付けるのに用います。

一行タグはタグもテキストも同じ行に書きます。
タグとテキストの間には半角スペースを入れて区別します。
タグとテキストの間のスペースは一つだけです。
二つめ以降はテキストの一部と認識されます。

例を見ましょう。

\begin{lstlisting}[caption=一行タグ]
:title 題名を書きます
:title  題名を書きます
\end{lstlisting}

先程の例から抜き出してきました。
1行目と2行目の違いは、タグの後ろにあるスペースの数です。
1行目はタグ名は「title」、そのテキストは「題名を書きます」となります。
2行目はタグ名は「title」、そのテキストは二つめの半角スペースを含む「 題名を書きます」となります。


\subsubsection{ブロックタグ}

次に\ref{enum:blocktag}.の「複数行分のテキストに名前を付けるタグ」についてですが、これも長いので\textbf{ブロックタグ}と呼ぶことにします。

ブロックタグは、改行を含むテキストデータに名前を付けるためのタグです。
たとえば、小説などの本文、あとがきなどに名前を付けるのに用います。

ブロックタグで名前付けされる範囲は説明が少々むずかしいので、先に例を出します。

\begin{lstlisting}[caption=ブロックタグ]
:body

 本文を書きます。
; この行はコメントです。無視されます。
 本文の二行目です。

:memo:

* 本文を書く
  * コメントを書く

* あとがきを書く
* それぞれ二行ずつ書くこと

:postscript

 あとがきを書きます。
 あとがき二行目。
 あとがきおわり。
\end{lstlisting}

ブロックタグは、改行を含むテキストデータに名前を付けるものだと言いました。
その範囲は\underline{タグの下の行から、次のタグの前の行まで}です。
ただし、\underline{前後の空行は無視}されます。

上の例でいえば、

\begin{itemize}
  \item bodyタグは「本文を書きます。」の行から「本文の二行目です。」の行まで
  \item memoタグは「\verb|*| 本文を書く」の行から「\verb|*| それぞれ二行ずつ書くこと」の行まで
  \item postscriptタグは「あとがきを書きます。」の行から「あとがきおわり。」の行まで
\end{itemize}

ということになります。

ここで「memoタグの後ろにもコロンがついている」のはミスではありません。
そのことは\ref{subsubsec:ignore}節で説明します。

ブロックタグ中では特殊な記法で以下のものを表現することができます:

\begin{enumerate}
  \item ルビ
  \item 強調(傍点を想定)
  \item 太字
  \item イタリック体(斜字)
  \item 太字かつ斜字
  \item 下線
  \item 見出し (4レベル)
  \item 引用 (4レベル)
\end{enumerate}

これらは難しいものではないので、使用例を見ればすぐに使い方がわかると思います。
\ref{subsec:example}節の例で出したsample.txtのbodyタグ内をもう一度見てみましょう。

\begin{lstlisting}[caption=ngn記法]
# 概要

 このテキストファイルはngn記法で書かれています。
 ngn記法では、#rb(読み)(ルビ)、#em(強調)(傍点を想定)、#bd(太字)、#it(斜字)、#bi(太字かつ斜字)、#ul(下線)を表現することができます

## 他の記法

 上記の他に、見出し(もう使っています)と、引用(これから使います)を使うことができます。

>" 初めに、神は天地を創造された。
 地は混沌であって、闇が深遠の#rb(面)(おもて)にあり、神の霊が水の面を動いていた。
 神は言われた。「光あれ。」こうして、光があった。"

\end{lstlisting}

ルビから下線までは、簡単ですね。
シャープ(\verb|#|)で始めて2文字のコマンド名を書いて、あとはルビや太字などにしたい文字列を括弧で囲むだけです。
コマンド名はそれぞれ英単語の二文字から取ったものなので、幾分覚えやすいと思います。

見出しと引用は少し特殊です。

\paragraph{見出し}

見出しは、章題などを表現するためのものです。
見出しは行頭にシャープ、半角スペースと続いたとき、それ以降の文字列が見出し文字列とみなされます。
pixivで「[chapter:なんとかかんとか]」というものがありますね。
あれとほとんど同じものです。

違う点は、レベルがあるという点です。
たとえば1節、1.1節、1.1.1節というふうに、章や節をもっと細かく分けたいときにレベルを使います。
見出しのレベルは先頭のシャープの数で決まり、最大が4までとなっています。
見出しの文字列中では、\underline{ルビや太字などといった記法は使えません}ので、注意してください。

\paragraph{引用}

引用は、書籍に載っている文章や、詩や手紙、はたまた作中の文章などを表現するものです。
ふつう引用は文章全体が地の文より何段も字下げされていますが、あれを表現します。
引用の範囲は、行頭に大なり記号(\verb|>|)、ダブルクォーテーション(\verb|"|)と続いたとき、次に現れるダブルクォーテーションまでをその範囲とします。

引用にもレベルがあります。
引用のレベルは、字下げの深さを表現するのに用います。
これは行頭の大なり記号の数で決まり、最大が4までとなっています。
この引用の中には、\underline{改行を入れる}こともできますし、\underline{ルビなど他の記法を使用する}ことができます。


\subsubsection{無視される行}
\label{subsubsec:ignore}

節の冒頭とブロックタグのところで言ったように、無視される行というのが存在します。
二種類があって、ひとつはブロックタグ中のコメント行、もうひとつではダミータグです。

ブロックタグの利用例を今一度見てみます。

\begin{lstlisting}[caption=ブロックタグの利用例]
:body

 本文を書きます。
; この行はコメントです。無視されます。
 本文の二行目です。

:memo:

* 本文を書く
  * コメントを書く
* あとがきを書く
* それぞれ二行ずつ書くこと

:postscript

 あとがきを書きます。
 あとがき二行目。
 あとがきおわり。
\end{lstlisting}

\paragraph{コメント}

ブロックタグ中、行頭がセミコロン(\verb|;|)で始まる行を\textbf{コメント}と言います。
コメントとした行は行末まで、ブロックタグ中の文字列だとみなされません。
コメント行は、行ごと無視されます。
本文中のいらなくなった行をとりあえずngnからは見えなくしてしまったり、そこに書いてある内容のメモをさらっと書くのに適しています。

\paragraph{ダミータグ}

ブロックタグの節で、ここにはbodyタグとmemoタグとpostscriptタグがあると書きました。
また、memoタグの後ろにコロンが付いているのはミスではないと述べました。

実は、memoタグは\textbf{ダミータグ}です。
\underline{ブロックタグの末尾にコロンを付ける}と、それはダミータグになります。
ダミータグは、ブロックタグでありながら、タグとしては無視される存在です。
上の例のように、本文とあとがきの間にメモ書きをしたいな、と思うことがあるかもしれません。
それらをコメント行としてすべての行の先頭にセミコロンをつけてもいいですが面倒です。
そのために、メモ書きなど\underline{複数行をブロックとして無視}するのがダミータグです。


\subsection{テンプレートファイルを用意する}

次に、テンプレートファイルをつくります。

テンプレートファイルをつくるときに行うことは、入力ファイルのタグをテンプレートファイル中のどの\underline{箇所に出力するのか指定する}ことです。

やり方はとても簡単です。
単にbodyタグだけを出力したい場合のテンプレートファイルは、次のようになります。

\begin{lstlisting}[caption=bodyタグのみが出力されるテンプレート]
#|body|#
\end{lstlisting}

タグ名を``\verb&#|&''と``\verb&|#&''で囲むだけです。
これを\textbf{マーカー}と呼ぶことにします。
もし入力ファイルに、マーカーに対応するタグがなかった場合は、マーカーがそのまま残ります。

一般的な使い方として、HTMLの簡易的なテンプレートを載せておきます。

\begin{lstlisting}[caption=HTMLのテンプレート]
<html lang="lang=ja">
<head>
<title>#|title|#</title>
</head>
<body>

<h1>#|title|#</h1>

#|body|#

<hr>

#|postscript|#

</body>
</html>
\end{lstlisting}


\subsection{ngnコマンドを実行して、出力ファイルを生成する}

最後に、出力ファイルを生成するためにngnコマンドを実行する方法を説明します。

ngnはコマンドラインアプリケーションです。
コマンドプロンプト、ターミナルなどのコマンドライン操作を多少知っている必要があります。
ここではngnコマンドの使い方を説明し、コマンドラインでの基本的な操作の説明はしません。

ngnコマンドへの引数の与え方は3種類あります。

\begin{lstlisting}[caption=ngnの引数]
$ ngn [テンプレートファイル] [入力ファイル] [出力ファイル]
$ ngn [テンプレートファイル] [入力ファイル]
$ ngn [テンプレートファイル]
\end{lstlisting}

コマンドの引数は最低でも1つ、最高で3つ必要です。
テンプレートファイルだけは必須です。
内部でngn記法の変換方法を決めるのに、テンプレートファイルの拡張子を使うからです。

最もシンプルなのが、引数を3つ与える呼び出し方でしょう。
テンプレートファイルと入力ファイル、出力ファイルを全て指定します。
このとき文字コードは、テンプレートファイルのもので出力されます。
出力ファイルに指定したファイルが存在した場合は、そのファイルに\underline{結果を上書きする}ということに注意してください。

2番目は、出力ファイルを省略した形です。
結果はどこに出力されるのかというと、標準出力に出力されます。
リダイレクトやパイプで何かに渡さない限り、ターミナル(あるいはコマンドプロンプト)に結果がつらつらと表示されるでしょう。

3番目、最後の形式は、テンプレートファイルのみを指定するものです。
標準入力から入力を読み取ってテンプレートファイルに埋め込み、標準出力へ結果を書き出します。
別のコマンドからパイプ等を駆使してデータを渡し、さらに別のコマンドへデータを流すような使い方ができるでしょう。


\section{レンダラについて}
\label{sec:renderer}

TODO:書け


\end{document}
