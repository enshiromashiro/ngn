\documentclass[a4j]{jsarticle}

\usepackage{listings, jlisting}
\lstset{%
breaklines=true,
frame=single,
basicstyle=\scriptsize
}


\newcommand{\emptyline}{\vspace{1em}}


\begin{document}

\title{ngn -novel page generator-}
\author{@subaru45}
\date{2015-02-22}
\maketitle


\section{概要}

\subsection{ngnってなんぞや}
ngnはテキストファイルとテンプレートから、HTMLファイルやpixiv小説形式のテキストファイルを生成する、簡単なテンプレートエンジンです。
テキストファイル中の部分に名前を付け(タグ付け)、タグ付けした部分をテンプレートファイル中に埋め込むことができます。

またngn独自の記法を用いてルビや太字、見出しや引用を最終的な形式(HTML、pixiv形式、などなど……)に依存しない形で記述しておき、テンプレートへの埋め込み時に最終的な形式での記法に変換することができます。

言葉ではわかりにくいと思うので、後に例を出します。


\subsection{だれ向けなのか}

一つのテキストファイルから、複数の形式にテンプレートを使って書き出したい人向けです。

たとえば、書いた小説データからウェブサイト用のHTML、pixiv用のテキストデータなどをコピペを駆使してつくって「めんどくせー!!」と思っている人には向いています。
あるいは、書き終わってから形式が決まる——最初からフォーマットが決まっていないという人にも向いているかもしれません。

逆に、常に一つの形式しか利用しない人、つまり、pixivにしか投稿しないので小説データにpixiv形式をベタ書きしている人などには、メリットがありません。


\subsection{利用例}


利用例を出します。
下記に示すように、ngn形式でテキストファイルを書いたとします。

\begin{lstlisting}[caption=sample.txt]
:title ngnについて
:author スバル
:description これは一行タグです

この文章は無視されます。

:body

# 概要

 このテキストファイルはngn記法で書かれています。
 ngn記法では、#rb(読み)(ルビ)、#em(強調)(傍点を想定)、#bd(太字)、#it(斜字)、#bi(太字かつ斜字)、#ul(下線)を表現することができます

## 他の記法

 上記の他に、見出し(もう使っています)と、引用(これから使います)を使うことができます。

>" 初めに、神は天地を創造された。
 地は混沌であって、闇が深遠の#rb(面)(おもて)にあり、神の霊が水の面を動いていた。
 神は言われた。「光あれ。」こうして、光があった。"

#it(旧約聖書 創世記 第1章 第1節〜第31節),
http://www12.ocn.ne.jp/~sokkidou/sokkig_01/0101_0131.html

; 行頭セミコロンはコメントです。この行と下の行は無視されます。
; なお、ngn記法で半角の括弧を利用しているため、
; 半角括弧の使用は避けたほうが良いです。

:memo:

これはメモなどを書く記法です。

:duplicate 同じタグが複数存在したら一番後ろにあるものが優先されます
:duplicate 重複タグ

\end{lstlisting}

また、テンプレートとして次のようなファイルを用意しておきます。

\begin{lstlisting}[caption=template.html]
<html lang="ja">
<head>
<meta http-equiv="Content-Type" content="text/html; charset=UTF-8">
<title>#|title|#</title>
<style type="text/css">
.emphasis {
  -webkit-text-emphasis: filled dot black;
  text-emphasis: filled dot black;
}
.bold {
  font-weight: bold;
}
.italic {
  font-style: italic;
}
.bold-italic {
  font-weight: bold;
  font-style: italic;
}
.underline {
  text-decoration: underline;
}
.quote1 {
  margin-left: 2em;
}
</style>

</head>

<body>
<h1>#|title|#</h1>
author: #|author|#<br>

#|body|#
<br>
descriptionタグの内容: #|description|#<br>
重複タグの内容: #|duplicate|#<br>

</body>
</html>
\end{lstlisting}

ここでコマンドラインで(ウィンドウで操作できるようにする予定です) $\$ngn template.html sample.txt output.html$ と叩くと、こんなファイルができあがります。

\begin{lstlisting}[caption=output.html]
<html lang="ja">
<head>
<meta http-equiv="Content-Type" content="text/html; charset=UTF-8">
<title>ngnについて</title>
<style type="text/css">
.emphasis {
  -webkit-text-emphasis: filled dot black;
  text-emphasis: filled dot black;
}
.bold {
  font-weight: bold;
}
.italic {
  font-style: italic;
}
.bold-italic {
  font-weight: bold;
  font-style: italic;
}
.underline {
  text-decoration: underline;
}
.quote1 {
  margin-left: 2em;
}
</style>

</head>

<body>
<h1>ngnについて</h1>
author: スバル<br>

<h2 class="header1">概要</h2>
<br>
 このテキストファイルはngn記法で書かれています。<br>
 ngn記法では、<ruby>読み<rp>(</rp><rt>ルビ</rt><rp>)</rp></ruby>、<span class="emphasis">強調</span>(傍点を想定)、<span class="bold">太字</span>、<span class="italic">斜字</span>、<span class="bold-italic">太字かつ斜字</span>、<span class="underline">下線</span>を表現することができます<br>
<br>
<h2 class="header2">他の記法</h2>
<br>
 上記の他に、見出し(もう使っています)と、引用(これから使います)を使うことができます。<br>
<br>
<p class="quote1">
 初めに、神は天地を創造された。<br>
 地は混沌であって、闇が深遠の<ruby>面<rp>(</rp><rt>おもて</rt><rp>)</rp></ruby>にあり、神の霊が水の面を動いていた。<br>
 神は言われた。「光あれ。」こうして、光があった。
</p>
<br>
<span class="italic">旧約聖書 創世記 第1章 第1節〜第31節</span>,<br>
http://www12.ocn.ne.jp/~sokkidou/sokkig_01/0101_0131.html<br>

<br>
descriptionタグの内容: これは一行タグです<br>
重複タグの内容: 重複タグ<br>

</body>
</html>
\end{lstlisting}

とまあ、こんなツールです。


\subsection{その他}

開発目的はウェブサイトの小説・文章ページを自動生成することですが、ほかの用途にも使えなくもないです。

出力形式はテンプレートファイルの拡張子で判別します。
これは独自に追加可能ですが、Common Lispの知識が必要です。
subaru45まで要望を寄せていだだければ、対応するかもしれません。

ngn は NYSL (煮るなり焼くなり好きにしろライセンス)\footnote{http://nysl.com} を採用しています。
NYSL の元で許される限りにおいて、如何様にも改変し配布し破棄することができます。


\section{導入}

プラットフォームによって導入方法は違います。

\begin{itemize}
\item Windows ユーザへ\\
  bitbucket の下記ダウンロードページ \\
  http://bitbucket.org/subaru45/ngn/downloads \\
から``win''の文字のついた一番新しいバージョンのファイルをダウンロードして、好きな場所に解凍してください。\\
  ウィンドウで操作できるようにする予定なので、少々おまちください。\\

\item OSX ユーザへ\\
  作者はMacを持っていないので、バイナリの提供ができません。Wineでなんとかならないかなあ。

\item *nix ユーザへ\\
  Ubuntu版バイナリなら提供できますが、そもそもいるのだろうか……。

\end{itemize}



\section{つかいかた}
ngn の使い方を説明します。

まず、出力したいテキストファイル中の文字列にタグを付けます。
タグが付されたテキストファイルを\textbf{タグ付きファイル}と呼びます。
タグ付きファイルにおいて、あるタグ spam が付けられた文字列を\textbf{spam のテキスト}と呼びます。
次にテンプレートを用意し、ファイル中に出力したいタグ指定子を記述しておきます。
最後に、タグ付きファイルとテンプレートを ngn に食わせると、テンプレートのタグ指定子の箇所に、そのタグのテキストが挿入されたファイルができあがります。

タグは書式さえ満たしていればいくつでも好きなように定義できます。
もしテンプレート中にタグ付きファイルにないタグを指定した場合、そのタグ指定子の箇所には空文字列が挿入されます。


\subsection{タグ付きファイル}
まず、タグを付けたテキストファイルを作成します。


タグには一行タグとブロックタグの二種類があります。
二種ともに、\textbf{行の先頭に}コロン ':' とタグ名を記述するという形式は同じです。
タグ名に利用できる文字はアルファベット小文字a-z、数字0-9、ハイフン'-'のみです。


\subsubsection{一行タグ}
一行タグは、改行を含まない文字列を記述するのに用います。
例えば、題名、著者名、日付などです。
\LaTeX でいうところの \textbackslash title や \textbackslash section 等のようなものだと考えてください (\LaTeX の上記コマンドは改行も含められますが……)。

一行タグの書式は以下です。
\begin{lstlisting}[caption=一行タグの書式]
:tag-name 文字列...
\end{lstlisting}
タグ名の後ろのスペースは一つです。
それ以降は改行までのスペースを含むすべての文字がタグ名に対応するテキストデータとなります。


\subsubsection{ブロックタグ}
ブロックタグは、改行を含む文字列を記述するのに用います。
例えば、本文、後書き、説明などです。
文書構造を記述するのに便利でしょう。

ブロックタグの書式は以下です。
\begin{lstlisting}[caption=ブロックタグの書式]
:tag-name[
文字列1...
文字列2...
...
文字列n...
:tag-name]
\end{lstlisting}
タグの開始・終了は、タグ名の後ろに括弧をつけて表します。
開始は '['、終了は ']' です。
カッコの直後で改行してください。
ブロックタグ内にタグの記述がされていた場合、タグではなくただの文字列として処理されます。


\subsubsection{注意事項}
タグ付きテキストファイル中では、現状、行頭でコロンを使用できません 
(エスケープシーケンス未実装のため)。

同名のタグが複数存在した場合、ファイルの先頭に近いものが保持され、それ以降の同名タグは無視されます。
例えば、下のタグ付きテキストファイル例の author タグのデータは「夏目漱石」となり「NATSUME Souseki」とはなりません。

\lstinputlisting[caption=タグ付きテキストファイル例]{sample/sample.txt}



\subsection{テンプレートファイル}
出力ファイルの雛形を作ります。
これは普通の HTML ファイルですが、タグのデータをどこに流し込むのか記述しておく必要があります。

テンプレートファイル中のタグの指定は
\begin{lstlisting}[caption=タグ指定の書式]
#|tag-name|#
\end{lstlisting}
の書式で行います。
これを書いた箇所がそのまま、そのタグのデータに置換されます。

もし指定したタグが存在しない場合、空文字列になります。
\lstinputlisting[caption=テンプレートファイルの例]{sample/template.html}



\subsection{ngn コマンド}
タグ付きテキストファイルとテンプレートファイルの用意ができたら、ファイルを生成します。
生成には ngn コマンドを用います。

\begin{lstlisting}[caption=ngn コマンドの使い方]
  ngn [input-filepath] [template-filepath]
\end{lstlisting}

[input-filepath] はタグ付きテキストファイルのパス、[template-filepath] はテンプレートファイルのパスです。
この二引数は必須です。
引数が足りない、または存在しないファイルだった場合はメッセージを吐いて何もせず終了します (たぶん)\footnote{その場合の終了コードは 1 だったはず}。

タグ付きテキストファイルの内容が残念だった場合の挙動はわかりません。
ちゃんと不正な入力は原因ごと表示するように、いつか改良します。

処理結果のファイルはカレントディレクトリに出力されます。
つまり、別のディレクトリに存在するファイルを引数に指定しても、処理結果はターミナルで ngn コマンドを実行したディレクトリに作成されます。
もし同名のファイルが存在した場合は \textbf{上書きせず、標準出力に}処理結果を出力します。
コマンドを叩いたらターミナルにHTMLなりがずらずら表れた場合は、同名ファイルが存在したということになります。

処理結果のファイル名は [input-filepath] の拡張子を [template-filepath] の拡張子に置き換えたものになります。
例えば、カレントディレクトリが /home/sora のとき、 [input-filepath] を /home/sora/text/wonder2.txt 、[template-filepath] を /home/sora/web/temp.html をそれぞれ引数として ngn を実行すると、出力ファイル名は wonder2.html となり、/home/sora に出力されます。

前2節でサンプルとしたファイルを引数に ngn を実行すると、次のような出力が得られます。

\lstinputlisting[caption=生成されたファイル]{sample/sample.html}


\end{document}
